\documentclass[10pt]{beamer}

\usetheme{metropolis}
\usepackage{appendixnumberbeamer}
\usepackage[english,russian]{babel}

\usepackage{booktabs}
\usepackage[scale=2]{ccicons}

\usepackage{pgfplots}
\usepgfplotslibrary{dateplot}

\usepackage{xspace}
\newcommand{\themename}{\textbf{\textsc{metropolis}}\xspace}

\setbeamercolor{background canvas}{bg=white}

\title{Типы данных. Подмножества. Векторизованные операции}
\subtitle{Язык программирования R}
\date{}
\author{Иркутский государственный университет}


\begin{document}
	\maketitle
	
	\begin{frame}[fragile]
		\frametitle{Типы данных в R}
		Язык R работает со следующими типами данных:
		\begin{itemize}
			\item[*] numeric – переменные, содержащие целочисленные значения (integer), действительные числа (double) и комплексные числа (complex);
			\begin{verbatim}
				
				double_value <- 290.7
				
				double_value <- 290
				
				integer_value <- 165L
				
				complex_value <- 8 + 5i
			\end{verbatim}	
		\end{itemize} 
	\end{frame} 
	
	\begin{frame}[fragile]
		\frametitle{Типы данных в R}
		\begin{itemize}
			\item[*] logical – переменные, содержащие логические значения: FALSE (сокращенно F) и TRUE (T);
			\begin{verbatim}
				
				is_monday <- F
				
				bool_t <- TRUE	
			\end{verbatim}	     
			\item[*] character – текстовые переменные (отдельные значения таких переменных задаются в двойных либо одинарных кавычках);
			
			\begin{verbatim}
				
				text_value <- "Hello, Word"
				
				char_value <- 'A'
				
			\end{verbatim}
		\end{itemize}    
	\end{frame}
	
	\begin{frame}[fragile]
		\frametitle{Векторы}
		Векторы – это одномерные объект, которые могут хранить числовые, текстовые или логические значения (комбинации не допускаются)
		\begin{verbatim}
			protein_id = c("INS", "GH1", "IL6", "TNF", "ACTB")
			
			expression_level = c(45.8, 12.3, 8.9, 6.7, 280.5),
			
			is_secreted = c(TRUE, TRUE, TRUE, TRUE, FALSE)
		\end{verbatim}
	\end{frame}
	
	\begin{frame}[fragile]
		\frametitle{Операции над векторами}
		\begin{verbatim}
			expression_sample1 <- c(15.2, 8.7, 23.4, 150.2)
			expression_sample2 <- c(18.5, 9.2, 25.1, 145.8)
			
			# Сложение - общая экспрессия
			total_expr <- expr_sample1 + expr_sample2
			
			# Вычитание - разница экспрессии
			diff_expr <- expr_sample1 - expr_sample2
			
			# Объединение
			all_expression <- c(expression_sample1, expression_sample2)
		\end{verbatim}
	\end{frame}
	
	
	\begin{frame}[fragile]
		\frametitle{Факторы}
		Фактор - специальный тип данных для представления порядковых и номинальных переменных с фиксированным набором возможных значений
		\begin{verbatim}
			expr_lvl <- factor(c("low", "high", "medium", "low", "high"),
			levels = c("low", "medium", "high"), 
			ordered = TRUE
			)
		\end{verbatim}
		где levels - это уникальные категории или возможные значения, которые может принимать фактор, ordered - флаг порядка категорий
	\end{frame}
	
	\begin{frame}[fragile]
		\frametitle{Операции над факторами}
		\begin{verbatim}
			# Изменить порядок уровней
			response <- factor(c("yes", "no", "yes", "maybe"))
			response <- factor(response, levels = c("no", "maybe", "yes"))
			
			# Поменять порядок
			response_rev <- factor(response, levels = rev(levels(response)))
			
			# Объединение факторов
			factor1 <- factor(c("A", "B", "A"))
			factor2 <- factor(c("B", "C", "A"))
			combined <- c(factor1, factor2)		
		\end{verbatim}
	\end{frame}
	
	
	\begin{frame}[fragile]
		\frametitle{Матрицы}
		Матрица – это двумерный массив данных, в котором все элементы имеют один и тот же тип (числовой, текстовый или логический). Матрицы создаются при помощи функции matrix. Общий синтаксис этой функции:
		\begin{verbatim}
			my_matrix <- matrix(
			вектор, 
			nrow=число строк, 
			ncol=число столбцов, 
			byrow=логическое значение, 
			dimnames=list(текстовый вектор с названиями строк, 
			текстовый вектор с названиями столбцов)
			)
		\end{verbatim}
		
		
	\end{frame}
	
	\begin{frame}[fragile]
		
		
		\frametitle{Массивы}
		Массивы — это объекты данных, которые могут хранить данные в более чем двух измерениях. Массивы создаются при помощи функции array:
		
		\begin{verbatim}
			my_array <- array(vector, dimensions, dimnames)
		\end{verbatim}
		где vector – это вектор с данными, dimensions – числовой вектор, определяющий размеры измерений, а dimnames – необязательный список названий измерений.	
	\end{frame}
	
	\begin{frame}[fragile]
		\frametitle{Таблицы данных}
		Таблицы данных – это основной класс объектов R, используемых для хранения данных.  Таблица данных создается при помощи функции data.frame():
		\begin{verbatim}
			my_data <- data.frame(col1, col2, col3, ...)
		\end{verbatim}
		где – col1, col2, col3, ... это векторы любого типа (текстового, числового или логического), которые станут столбцами таблицы.	
	\end{frame}
	
	\begin{frame}[fragile]
		
		
		\frametitle{Пример таблицы данных}
		
		\begin{verbatim}
			genome_data <- data.frame(
			gene_id = c("BRCA1", "TP53", "EGFR", "HBB"),
			chromosome = c("17", "17", "7", "11"),
			start_position = c(43044295, 7668402, 55019017, 5248232),
			end_position = c(43170245, 7687550, 55211628, 5249264),
			strand = c("+", "-", "+", "-"),
			expression = c(15.2, 8.7, 23.4, 150.2)
			)
		\end{verbatim}	
	\end{frame}
	
	\begin{frame}[fragile]
		\frametitle{Операции с таблицами данных}
		rbind() - объединяет два набора данных по вертикали (по строкам):
		{\fontsize{7}{8}\selectfont	
			\begin{verbatim}
				# Первый набор данных - экспрессия генов
				gene_expression <- data.frame(
				gene_id = c("TP53", "BRCA1", "EGFR"),
				sample_A = c(15.2, 8.7, 25.4),
				sample_B = c(18.9, 7.3, 28.1)
				)
				
				# Второй набор данных - дополнительная экспрессия
				dop_expression <- data.frame(
				gene_id = c("AKT1", "PTEN"),
				sample_A = c(22.3, 9.8),
				sample_B = c(24.1, 8.9)
				)
				
				# Объединяем два набора данных по экспрессии генов
				all_expression <- rbind(gene_expression, more_expression)
				
				gene_id sample_A sample_B
				1    TP53     15.2     18.9
				2   BRCA1      8.7      7.3
				3   EGFR     25.4     28.1
				4   AKT1     22.3     24.1
				5   PTEN      9.8      8.9		
			\end{verbatim}
		}	
		
	\end{frame}
	
	
	\begin{frame}[fragile]
		\frametitle{Операции с таблицами данных}
		cbind() - объединяет две матрицы или таблицы по горизонтали (по столбцам):
		{\fontsize{7}{8}\selectfont	
			\begin{verbatim}
				# Первый набор данных - экспрессия генов
				gene_expression <- data.frame(
				gene_id = c("TP53", "BRCA1", "EGFR"),
				sample_A = c(15.2, 8.7, 25.4),
				sample_B = c(18.9, 7.3, 28.1)
				)
				
				# Добавляем данные по третьему образцу
				sample_C_data <- data.frame(
				sample_C = c(16.8, 9.2, 26.7)
				)
				
				# Объединяем с исходными данными
				expression_with_C <- cbind(gene_expression, sample_C_data)
				
				gene_id sample_A sample_B sample_C
				1    TP53     15.2     18.9     16.8
				2   BRCA1      8.7      7.3      9.2
				3   EGFR     25.4     28.1     26.7		
			\end{verbatim}
		}	
		
	\end{frame}
	
	\begin{frame}[fragile]
		\frametitle{Операции с таблицами данных}
		merge() - объединяет таблицы по ключу:
		{\fontsize{7}{8}\selectfont	
			\begin{verbatim}
				# Таблица с экспрессией генов
				expression_data <- data.frame(
				gene_id = c("TP53", "BRCA1", "EGFR", "MYC", "AKT1"),
				expression = c(15.2, 8.7, 25.4, 12.1, 18.9),
				sample = "sample_A"
				)
				
				# Таблица с мутациями
				mutation_data <- data.frame(
				gene_id = c("TP53", "BRCA1", "EGFR", "PTEN"),
				mutation_count = c(3, 1, 2, 1),
				mutation_type = c("missense", "frameshift", "nonsense", "missense")
				)
				
				# Общие гены
				inner_join <- merge(expression_data, mutation_data, by = "gene_id")
				  gene_id expression   sample mutation_count mutation_type
				1   BRCA1        8.7 sample_A              1    frameshift
				2    EGFR       25.4 sample_A              2      nonsense
				3    TP53       15.2 sample_A              3      missense	
			\end{verbatim}
		}	
		
	\end{frame}
	
	\begin{frame}[fragile]
		\frametitle{Списки}
		Cписок – это упорядоченная коллекция объектов. Список может объединять разные объекты вне зависимости от их типов. К примеру, список может одновременно содержать векторы, матрицы, таблицы данных и другие списки.
		\begin{verbatim}
			my_list <- list(объект1, объект2, ...)
			
			gene_info <- list(
			gene_name = "TP53",
			chromosome = 17,
			position = c(7668402, 7687550),
			)
		\end{verbatim}
	\end{frame}
	
	
	\begin{frame}[fragile]
		\frametitle{Чтение табличных данных}
		Функция read.table() позволяет импортировать данные из текстовых файлов с разделителями. Некоторые параметры этой функции:
		{\fontsize{6}{7}\selectfont
		\begin{verbatim}
file - путь к файлу
			
header - логическое значение, указывает наличие строки с названиями колонок в начале файла
			
nrows - максимальное количество строк для чтения (-1 означает все строки)
			
sep - символ-разделитель колонок в файле
			
dec - символ, используемый как десятичный разделитель в числах
			
skip - количество строк в начале файла для пропуска перед чтением данных
			
na.strings - вектор строк, которые интерпретируются как пропущенные значения (NA)
			
colClasses - вектор, определяющий классы данных для каждой колонки
			
stringsAsFactors - логическое значение, указывает должны ли строковые колонки преобразовываться в факторы
		\end{verbatim}
	}
	\end{frame}
	
	
	\begin{frame}[fragile]
		\frametitle{Пример использования функции read.table()}
		\begin{verbatim}
			data <- read.table("data.txt", 
			header = TRUE,        # есть заголовок
			sep = "\t",           # разделитель - табуляция
			stringsAsFactors = FALSE)  # не преобразовывать строки в факторы
			
			# Параметры по умолчанию (для CSV)
			data <- read.table("data.csv", 
			header = TRUE,
			sep = ",",
			dec = ".")
		\end{verbatim}
	\end{frame}
	
	
	\begin{frame}[fragile]
		\frametitle{Подмножества}
		Квадратные скобки [ — сохраняют класс исходного объекта при извлечении подмножества и поддерживают выбор произвольного количества элементов.
		
		Двойные квадратные скобки [[ — предназначены для извлечения единичного элемента из списка или таблицы данных, при этом возвращаемый объект может иметь другой класс.
		
		Оператор \$ — обеспечивает доступ к элементам списка или таблицы данных по имени, аналогично работе оператора [[.
		
		
		
	\end{frame}
	

	
	\begin{frame}[fragile]
		\frametitle{Подмножества}
		Подмножества данных можно выбирать на основе любой структуры данных. Например, на основе векторов:
		{\fontsize{8}{9}\selectfont
			
			
				\begin{verbatim}
gene_expression <- c(15.2, 8.7, 23.4, 150.2, 5.3)
names(gene_expression) <- c("BRCA1", "TP53", "EGFR", "HBB", "CFTR")
					
# Выборка по индексам
gene_expression[3]           # EGFR: 23.4
gene_expression[c(2,4)]      # TP53: 8.7, HBB: 150.2
					
# Выборка по условиям
gene_expression[gene_expression > 20]    # гены с экспрессией > 20
gene_expression[gene_expression < 10]   # гены с экспрессией < 10
					
# Выборка по именам генов
gene_expression[c("BRCA1", "HBB")]       # конкретные гены\end{verbatim}
			
}
		
		
	\end{frame}
	
	\begin{frame}[fragile]
		\frametitle{Подмножества}
		Подмножества данных также можно выбирать на основе матриц:
		{\fontsize{8}{9}\selectfont
			
				\begin{verbatim}
expr_matrix <- matrix(
    c(25.4, 18.9, 32.1, 45.2, 12.8, 8.7, 150.8, 95.3, 120.5), 
    nrow = 3,
    byrow = TRUE,
    dimnames = list(
    c("SYP", "GAD1", "GFAP"), 
    c("Cortex", "Cerebellum", "Hippocampus")))
					
marker_genes <- expr_matrix[c("SYP", "GFAP"), ]
custom_subset <- expr_matrix[c("GAD1", "GFAP"), c("Cerebellum", "Hippocampus")]
high_gfap_regions <- expr_matrix["GFAP", expr_matrix["GFAP", ] > 100]          
					
				\end{verbatim}
		}
		
		
	\end{frame}
	
	\begin{frame}[fragile]
		\frametitle{Подмножества}
		Подмножество на основе таблицы данных:
		{\fontsize{7}{7}\selectfont
				\begin{verbatim}
gene_data <- data.frame(
    gene_name = c("TP53", "BRCA1", "EGFR", "MYC", "AKT1", "PTEN"),
    expression = c(15.2, 8.7, 25.4, 12.1, 18.9, 9.5),
    chromosome = c("chr17", "chr17", "chr7", "chr8", "chr14", "chr10"),
    is_oncogene = c(FALSE, FALSE, TRUE, TRUE, TRUE, FALSE),
    mutation_count = c(3, 1, 2, 0, 1, 2)
)
					
# 1. Подмножество по строкам и столбцам - Строки 2-4, колонки gene_name и expression
subset1 <- gene_data[2:4, c("gene_name", "expression")]
					
# 2. Подмножество по условию (высокая экспрессия)
high_expression <- gene_data[gene_data$expression > 15, ]
					
# 3. Подмножество по нескольким условиям (онкогены с мутациями)
sample1 <- gene_data[gene_data$is_oncogene & gene_data$mutation_count > 0, ]
					
# 4. Подмножество с оператором ИЛИ
sample2 <- gene_data[gene_data$chromosome == "chr17" | gene_data$expression > 20, ]
					
# 5. Подмножество по индексам строк
selected_rows <- gene_data[c(1, 3, 5), ]   \end{verbatim}
		}
		
		
	\end{frame}
	
	\begin{frame}[fragile]
		\frametitle{Подмножества}
		Функция subset() – позволяет выбрать подбвыборку данных на основе какого-либо условия.
		{\fontsize{8}{9}\selectfont
			
				\begin{verbatim}
					chr17_high_expr <- subset(genome_data, 
					chromosome == "17" & expression > 10)
					
					low_expr_genes <- subset(genome_data, 
					expression < 10, 
					select = c(gene_id, chromosome, expression))\end{verbatim}
			
		}
		
		
	\end{frame}
	
	
	\begin{frame}[fragile]
		\frametitle{Обработка пропущенных значений}
		В R NA (Not Available) означает отсутствие значения, тогда как NaN (Not a Number) указывает на неопределенный результат числовой операции, например, \(0/0\). NA может применяться к любым типам данных (числовым и символьным) и означает, что данные были, но не были измерены или записаны. NaN возникает только в числовых вычислениях и обозначает результат, который нельзя представить как число. 
		
		is.na() - проверка на пропущенные значения
		
		is.nan() - проверка на "не число" (Not a Number)
		{\fontsize{7}{7}\selectfont
		\begin{verbatim}
			num_vector <- c(1, 2, NA, 4)
			is.na(num_vector)  # FALSE FALSE  TRUE FALSE
			char_vector <- c("a", "b", NA, "d")
			is.na(char_vector)  # FALSE FALSE  TRUE FALSE
			log_vector <- c(TRUE, FALSE, NA, TRUE)
			is.na(log_vector)  # FALSE FALSE  TRUE FALSE
			
			z <- c(1, 2, NA, NaN, 5)
			print(is.na(z))  # FALSE FALSE  TRUE  TRUE FALSE
			print(is.nan(z))  # FALSE FALSE FALSE  TRUE FALSE
	    \end{verbatim}	}
	\end{frame}
	
	
	
	
	\begin{frame}[fragile]
		\frametitle{Векторизованные операции}
		Векторизация — это возможность применять операции ко всему вектору сразу, без использования явных циклов. Это одна из ключевых особенностей R, которая делает код эффективным и читаемым.:
		{\fontsize{8}{9}\selectfont
		\begin{verbatim}
			# Векторы
			x <- 10:14
			y <- 20:24
			
			print(x + y)
			print(x > 12)
			print(y * 2 + x * 3)
			print(x %% 2 == 0)
			
			# Матрицы
			mat1 <- matrix(1:9, nrow = 3)
			mat2 <- matrix(10:18, nrow = 3)
			
			print(mat1 + mat2)       # Поэлементное сложение
			print(mat1 * 2 + mat2)   # Умножение на скаляр
			print(mat1 * mat2)       # Поэлементное умножение (не матричное!)   
		\end{verbatim}
	}
	\end{frame}
	
	
	
	
\end{document}
