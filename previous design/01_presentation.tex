\documentclass{beamer}
\usetheme{default}

\usepackage{listings}
\usepackage[english,russian]{babel}


\title{Подмножества. Управляющие конструкции. Функции. 
	Область видимости. Дата и время
}
\author{TeXstudio Team}
\begin{document}
	
	
\begin{frame}[plain]
    \maketitle
\end{frame}

\begin{frame}[fragile]
    \frametitle{Типы данных в R}
    Язык R работает со следующими типами данных:
    
    \begin{itemize}
    	\item[*] numeric – переменные, содержащие целочисленные значения (integer), действительные числа (double) и комплексные числа (complex);
    \begin{verbatim}
    
    double_value <- 290.7
    	
    double_value <- 290
    
    integer_value <- 165L
    
    complex_value <- 8 + 5i
    \end{verbatim}
      
    

   \end{itemize} 
\end{frame}   
\begin{frame}[fragile]
	\frametitle{Типы данных в R}
	\begin{itemize}
	\item[*] logical – переменные, содержащие логические значения: FALSE (сокращенно F) и TRUE (T);
		\begin{verbatim}
			
		is_monday <- F
		
		bool_t <- TRUE	
		\end{verbatim}	     
    \item[*] character – текстовые переменные (отдельные значения таких переменных задаются в двойных либо одинарных кавычках);
    
        \begin{verbatim}
        	
    text_value <- "Hello, Word"
    
    char_value <- 'A'
    
		\end{verbatim}
	\end{itemize}    
\end{frame}

\begin{frame}[fragile]
	\frametitle{Векторы}
	Векторы – это одномерный объект, которые могут хранить числовые, текстовые или логические значения (комбинации не допускаются)
	\begin{verbatim}
		a <- c(1, 2, 5, 3, 6, -2, 4)
		
		b <- c("one", "two", "three", "four")
		
		c <- c(TRUE, FALSE, TRUE, TRUE, FALSE)
	\end{verbatim}
\end{frame}

\begin{frame}[fragile]
	\frametitle{Матрицы}
Матрица – это двумерный массив данных, в котором все элементы имеют один и тот же тип (числовой, текстовый или логический). Матрицы создаются при помощи функции matrix. Общий синтаксис этой функции:
\begin{verbatim}
mymatrix <- matrix(
    вектор, 
    nrow=число строк, 
    ncol=число столбцов, 
    byrow=логическое значение, 
    dimnames=list(текстовый вектор с названиями строк, 
    текстовый вектор с названиями столбцов)
)
\end{verbatim}


\end{frame}

\begin{frame}[fragile]
	\frametitle{Подмножества}
	Функция subset() – позволяет выбрать подбвыборку данных на основе какого-либо условия. 
	
	Вот два примера:
	{\fontsize{8}{9}\selectfont
\begin{verbatim}
		
	newdata <- subset(leadership, age >= 35 | age < 23, select=c(q1, q2, q3, q4))
	            
	newdata <- subset(leadership, gender=="M" \& age > 25, select=gender:q4)
\end{verbatim}}	
\end{frame}

\begin{frame}
	\frametitle{Управляющие конструкции}
	Разновидности условных конструкций:
	\begin{itemize}
	\item[*]if условный оператор if-else выполняет инструкцию, если заданное условие верно. Также есть возможность выполнить другую инструкцию, если условие не верно.
	
	\item[*]ifelse - компактная и векторизованная версия оператора if-else.
	
	\item[*]switch - оператор выбора switch выбирает инструкцию для выполнения в зависимости от значения выражения expr. Он имеет следующий синтаксис: switch(expr, ...) где многоточие (...) означает инструкции, соответствующие возможным значениям expr
	\end{itemize}
\end{frame}	

\begin{frame}{Числовые и текстовые функции}
	Математические функции:
	\resizebox{\textwidth}{!}{
	\begin{tabular}{|p{4cm}|p{10cm}|}  % l-влево, c-по центру, r-
	
		\hline
		abs(x) & Модуль \\
				\hline
		sqrt(x) & Квадратный корень \\
				\hline
		ceiling(x) &  Ближайшее целое число, не меньшее, чем x \\
				\hline
		floor(x) &  Ближайшее целое число, не большее, чем x \\
				\hline
		trunk(x) & Целое число, полученное округлением x в сторону нуля \\
				\hline
		round(x, digits=n) & Округляет x до заданного числа знаков n после запятой \\
				\hline
		signif(x, digits=n) & Округляет x до заданного числа n значащих цифр \\
				\hline
	
	\end{tabular}
}
\end{frame}

\begin{frame}
	\frametitle{Числовые и текстовые функции}
	
	Статистические функции:
	
	\resizebox{\textwidth}{!}{
		\begin{tabular}{|p{4cm}|p{10cm}|}  % l-влево, c-по центру, r-
			
			\hline
			mean(x) & Среднее арифметическое \\
			\hline
			median(x) & Медиана \\
			\hline
			sd(x) &  Стандартное отклонение \\
			\hline
			var(x) & Дисперсия \\
			\hline
			quantile(x, probs) & Квантили, где x – числовой вектор, для которого вычисляются квантили, а probs – числовой вектор с вероятностями в диапазоне [0; 1] \\
			\hline
			range(x) & Размах значений \\
			\hline
			sum(x) & Сумма \\
			\hline
			
		\end{tabular}
}		
	
	
\end{frame}




\begin{frame}[fragile]
		\frametitle{Пользовательские функции}
		\begin{verbatim}
		y <- 10  # Глобальная переменная
		
		my_function <- function(x) {
			result <- x + y  # y находится в глобальной области!
			return(result)
		}
		
		my_function(5)  # → 15
		\end{verbatim}
\end{frame}

\begin{frame}[fragile]
	\frametitle{Область видимости}
	\begin{verbatim}
		x <- "глобальная"
		
		outer_func <- function() {
			x <- "внешняя"
			
			inner_func <- function() {
				x <- "локальная"
				print(x)  # Найдет "локальную"
			}
			inner_func()
		}
	\end{verbatim}
\end{frame}

\begin{frame}[fragile]
	\frametitle{Дата и время}
	\begin{verbatim}
	     as.Date(x, "input_format")
    \end{verbatim} 	
	где x – это дата в текстовом формате, а input\_format определяет формат представления даты
\end{frame}

\begin{frame}[fragile]
	\frametitle{Циклы}
	Конструкция цикла for выполняет инструкцию statement для каждого значения в последовательности seq. Она имеет следующий синтаксис:
	\begin{verbatim}
	for (var in seq) {
	    statement
	}
    \end{verbatim}
	Следующий код:
	\begin{verbatim}
	for (i in 1:5)  {
	    print("Hello World")
	}
\end{verbatim}
	выведет строку Hello World 5 раз.
\end{frame}

\begin{frame}[fragile]
	\frametitle{Циклы}
	Конструкция цикла while повторно выполняет инструкцию, пока заданное условие остается истинным. Она имеет следующий синтаксис:
	\begin{verbatim}
	while (cond) {
		statement
	}
    \end{verbatim}
	Следующий код:
\begin{verbatim}	
	i <- 5
	while (i > 0) {
		print("Hello World"); 
		i <- i - 1
	}
\end{verbatim}	
	тоже выведет строку Hello World 5 раз.
\end{frame}



\begin{frame}
	\frametitle{Название}
\end{frame}


\end{document}
